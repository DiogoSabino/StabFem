\section{Chapterexample}

\begin{enumerate}
\item a exmaple item:

  \begin{itemize}[noitemsep]
  \item Record the \emph{ID} and \emph{name} of a student on each line.
  \item Use a comma (\texttt{,}) to separate the ID and name on each line.
  \item Leave an \emph{empty line} between groups.
  \end{itemize}
  
\item Save file using \texttt{.txt} or \texttt{.csv} extension, e.g.\\\texttt{DIT1234-Apr2013-groups.txt}.

\medskip

\begin{leftbar}
You may also do this You may also do this You may also do this You may also do thisYou may also do thisYou may also do thisYou may also do thisYou may also do this.
\end{leftbar}




\item Launch the Stabfem tool by double-clicking on its icon.
\item From the menu bar, select \menu{File>Import...} or press \keys{Ctrl+I}.
\item Select a \emph{group list file} (\texttt{*.txt, *.csv}) to import from, e.g.\\\texttt{DIT1234-Apr2013-groups.txt}.
\item Stabfem will then ask you for a \emph{new} file name to save the new score list as (\texttt{*.xml}), e.g.~\texttt{DIT1234-Apr2013-Assgn1.xml}.

\medskip

\begin{leftbar}
Make sure the \texttt{.xml} score filename selected does not already exist. Otherwise the existing score file \textbf{may be overwritten without warning}!
\end{leftbar}

\medskip

\item Stabfem imports the group lists and displays the student IDs and names in the main panel (Figure~\ref{fig:grouplist}).


\end{enumerate}

